\documentclass[12pt]{article}

% fonts
\usepackage[scaled=0.92]{helvet}   % set Helvetica as the sans-serif font
\renewcommand{\rmdefault}{ptm}     % set Times as the default text font

% db/ not mandatory, but i recommend you use mtpro for math fonts.
% there is a free version called mtprolite.

% \usepackage[amssymbols,subscriptcorrection,slantedGreek,nofontinfo]{mtpro2}

\usepackage[T1]{fontenc}
\usepackage{amsmath}
\usepackage{amsfonts}

% page numbers
\usepackage{fancyhdr}
\fancypagestyle{newstyle}{
\fancyhf{} % clear all header and footer fields
\fancyfoot[R]{\vspace{0.1in} \small \thepage}
\renewcommand{\headrulewidth}{0pt}
\renewcommand{\footrulewidth}{0pt}}
\pagestyle{newstyle}

% geometry of the page
\usepackage[top=1in,
            bottom=1in,
            left=1in,
            right=1in]{geometry}

% paragraph spacing
\setlength{\parindent}{0pt}
\setlength{\parskip}{2ex plus 0.4ex minus 0.2ex}

% useful packages
\usepackage{natbib}
\bibliographystyle{dinat}
\usepackage{epsfig}
\usepackage{url}
\usepackage{bm}

\usepackage{titlesec}
\titleformat*{\section}{\normalsize\bfseries}
\titleformat*{\subsection}{\normalsize\bfseries}

\usepackage{enumitem}
\setlist{nolistsep}

\begin{document}

\begin{flushleft}
\textbf{Propagator BBVI: a flexible framework for rapid design of BBVI
  optimization algorithms} \\
Adam Massmann (akm2203) \\
\today
\end{flushleft}

Here I present propagator BBVI: a BBVI inference approach built on top
of a propagator computing framework. Propagators consist of stateful
cells that accumulate information about a value, and autonomous
machines that add information about a value. In the case of BBVI,
cells are each variational factor, and the autonomous machines are the
computations that calculate gradients from the joint and update the
variational families. This framework allows us to rapidly develop and
test many (possibly novel) stochastic gradient update schemes: each
variational factor can use different step size functions and
convergence specifications, and we can mix both re-parameterization
and score-based gradient estimates on portions of the model that are
differentiable or not. Additionally, the propagator based framework
has the potential to scale to massively parallel computation,
implicitly with no necessary specification from the user. I use
propagator BBVI to test the convergence properties of different
algorithm designs: specifically I examine the usefulness of
iteratively optimizing some portions of the posterior with others held
fixed (e.g. a sort of black box CAVI). Propagator BBVI allows the user
to easily and rapidly specify many such algorithm designs.


\section{Introduction}

% goals of library
Box's loop allows us to interatively refine assumptions about our
data. However, efficient movement around the loop requires that
computationally tractable inference algorithms are easy to derive from
a model encoding our assumptions. Black box variational inference
(BBVI) performs inference on a very broad set of probablistic models
without difficult model-specific derivations, and is a path towards
quick iterations over Box's loop \citep[e.g.,][]{ranganath-2014,
  kuc-2017}. Here I build a new approach to BBVI upon a propagator
computing framework. The \texttt{propagator-bbvi} Haskell library aims for:

\begin{itemize}
\item A balance between the ease of specification and the flexibility
  of the inference optmizition: the probablistic model should be easy
  to specify on the computer and BBVI inference should follow
  automatically from the specification. Howev er, the library is also
  designed around easy user modification and performance tuning of the
  optmization algorithms underlying the inference methods. So, there
  is a tradeoff between automatic inference from the model
  specification and the flexibility to tune the inference algorithm
  for tractability. Here we favor a balance between adding complexity
  in what the user must specify in order to run inference, and
  granting the user the ability to try many different optimization
  strategies underlying inference. Sensible defaults can mitigate
  someof the added complexity of user specification.
\item Extensibility: the library should be easy for users to
  extend. For example, adding new variational distributions or
  optimization algorithms should be simple and additive to the
  codebase: no need to rewrite or modify existing code. To achieve
  this goal, we favor a shallow library structure rather than a more
  deeply embedded domain specific language (DSL). The benefits of a
  shallow library are that it gives users more flexibility for
  extension and specification through a designed blending between the
  library and exisitng language abstractions (in this case Haskell)
  . However, the downsides are that a shallow embedding requires that
  users may need a deeper understanding for how the library works
  internally, as well as familiarity with Haskell. A library approach
  could be a nice alternative to popular general inference tools like
  Stan and Infer.NET that favor more of a DSL approach (but note I am
  not as familiar with tools like Edward and Pyro; they may take a
  different approach).
\item Ability to scale to massive data and models: inference methods
  should scale to massive data. In addition to previous work on
  stochastic gradient methods to scale to massive data
  \citep[e.g.,][]{hoffman-2013, ranganath-2014, kuc-2017}, we hope to
  build a framework that easys parallelization of inference
  algorithms. In keeping with the extensibility goal, development of
  new parallel scheduling and data subsampling techniques possible
  without significant modification of the larger codebase, most
  importantly the machinery for specifiying models.
\end{itemize}

As I describe the library design and its novelty for developing new
approaches to BBVI, I will refer back to these three goals:
\textbf{\emph{optmiziation flexibility}}, \textbf{\emph{extensibility}}, and \textbf{\emph{scalability}}.

\section{Introduction to propagators}

The propagator computing framework \citep{radul-2009} consists of
stateful cells that accumulate information about a value and stateless
autonomous propagators that do computations and add information to
cells. In other words, cells do no computation and propagators store
no data. The user attaches propagators to cells. Their are two
attachment types for propagators: \textit{watch} and \textit{write}
attachments. When cells acquire new information, they trigger any
propagators watching them to execute their computations with the new
data, and those propagators will write the result of their
computations to any cells they have a \textit{write} attachment to. If any
of these writes are new information for a given cell, then they will
trigger their own watching propagators to fire. In this manner,
propagators and cells form an asycnrounous computation network that
runs until no cells in the network can aquire any more information
(``quiescence'').

The propagator model is useful in many computational applications,
including functional reactive programming, dependency directed search,
and constraint satisfaction. For a full discussion I recommend
\citet{radul-2009}. However, why propagators for BBVI? In short, they
are a path towards massively parallel computation. Because all
computation is asyncronous and there is no global state, the ordering
and scheduling of computation is completely independent of the
propagator network. So, once a problem is coded in terms of
propagators, one can alter the scheduling of the propagator
computations to suit their goals without touching the propagator
network. In this way propgators directly lend themselves to my design
goals: (1) \textit{scalability} through  parralel comutation,
and (2) \textit{extensibility}: computation scheduling is independent
of the algorithm code, so new parallel schedulers can be developed
without modification of any other pieces of the library.

\section{BBVI with Propagators}

Variational factors are represented as cells: they accumulate
information moving them closer (on average) to the true
posterior. Computations calculating gradients and updating the
variational factors are the propagators. A deeper dive into the design
will illuminate the consequences and benefits of this new approach to
BBVI. The library is publically available at:
\url{https://github.com/massma/propagator-bbvi}, and some sparse (but
growing) documentation is available at:
\url{http://www.columbia.edu/~akm2203/propagator-bbvi-0.1.0.0/}.

\subsection{Cells: variational factors}

Variational factor cells contain three pieces of information: the
variational factor and its parameterization (e.g. for a Gaussian, the
mean and variance), a \textit{time} integer correspnding to the number
of times the variational factor has been updated, and a
\textit{memory} vector equal in length to the number of parameters of
the variational factor (e.g. in the case of the Gaussian, a vector of
length 2). The \textit{time} and \textit{memory} data are useful for
the gradient updates: propagators calculating gradients can use these
data to generate a step size for stochastic gradient optmization. For
example, in \citet{kuc-2017} the step size parameterization uses both
the number of gradient update interations, as a quantity determined by
the history of gradient mangitudes. By including \textit{time} and
\textit{memory} locations in our variational factor cells, our
gradient update propagators can employ such step size
parameterizations. Step size paramterizations that do not require
memory information need not use these data, but they are still
available. Inclusion of \textit{time} and \textit{memory} aid our goal
of optimization flexibility.

In addition to the information stored in variational factor cells,
invariant data are also defined for each variational factor. These
include the prior on the variational factor's latent variable, the
step size function that takes the cell's \textit{time} and
\textit{memory} data and calcualtes a step size and new
\textit{memory} parameter, and a count of the total number of factors
in the model's likelihood function that contain the latent variable
corresponding to the variational factor. This count will be elaborated
on in Section \ref{Propagators}, but it facilitates distributed
updates by many propagators to a single variational factor. This and
the step size function aid optmization flexibility: they user can
easily specify many different optmization approaches to a factor
independent of thsoe used in for all other variational factors.

\subsubsection{Variational factor representation}

A variational factor's parameters can be stored in any form. For
example, in \texttt{propagator-bbvi}'s current version the Dirichlet
distribution's paramters are stored as a vector, while the normal
distribution's parameters are stored in a record datatype. However,
all variational factors must implement the following functions to be a
part of the necessary typeclasses for inference within
\texttt{propagator-bbvi}\footnote{a quick rundown on Haskell syntax in
  case you are not familiar: ``::'' means ``has type.'' so
  ``fromParamVector :: ParamVector -> a'' means the function
  ``fromParamVector'' has a type which takes in a parameter vector and
  outputs a distribution of arbitrary type ``a.'' But don't worry too
  much about it; I will explain all code snippits in normal language.}:

\begin{verbatim}
class DistUtil a where
  fromParamVector :: ParamVector -> a
  toParamVector :: a -> ParamVector
  nParams :: a -> Int

class DistUtil a => Dist a c where
  resample :: a -> GenST s -> ST s c
  logProb :: a -> c -> Double
  paramGradOfLogQ ::  a -> c -> ParamVector
\end{verbatim}

The typeclass \texttt{DistUtil} provides functions that help with the
necessary vector arithmetic for calculating gradients:
\texttt{fromParamVector} converts a vector of paramters to a
distribution, \texttt{toParamVector} converts a distribution to a
vector of its parameters, and nParams takes a distribution and
provides the number of parameters it containts. For a Gaussian
example, \texttt{nParams} would return 2, and \texttt{fromParamVector}
would read a vector containing the mean and variance and return a
Gaussian, while \texttt{toParamVector} would take a Gaussian and
return a vector containing the mean and variance. These functions are
generally trivial to implement.

The typeclass \texttt{Dist} provides the necessary functions to
compute score-gradients \citep{ranganath-2014}. \texttt{resample}
takes a distribution and a random number generator, and returns a
sample from teh distribution. \texttt{lobProb} takes a distribution
and an observation and return the log of the probability/desity of the
observation. \texttt{paramGradOfLogQ} calculates the gradient of the
log probability of an observation under a distribution, with repect to
the parameters of the distribution. The gradient must be in vector
form, and the locations of the gradients within the vector should
match the location of the parameters in \texttt{fromParamVector} and
\texttt{toParamVector}. This typeclass approach to representing
variational factors aids the design goal of extensibility: to add a
new variational factor for use in \texttt{propagator-bbvi}, one need
only define six functions, three of which are trivial.

However, one may also wish to use a reparameterization gradient
updates \citep{kuc-2017} for a variational factor. Any distribution
that implments the following four functions can be used with
reparameterization propagators:

\begin{verbatim}
class (Dist a c) => Differentiable a c where
  epsilon ::  a -> GenST s -> ST s c
  transform :: a -> c -> c
  gradTransform :: a -> c -> ParamVector
  sampleGradOfLogQ ::  a -> c -> c
\end{verbatim}

\texttt{epsilon} generates an un-transformed sample: note that while
the function takes a distribution as input, this is only to keep the
type system happy. \texttt{transform} transforms a sample generated
from $\epsilon$ into a sample from the variational
distribution. \texttt{gradTransform} takes the gradient of the
transformation with respect to the transformation paramters, while
\texttt{sampleGradOfLogQ} takes the gradient of the log-probability
function with respect to a transformed sample. For example, in the
Gaussian example if \texttt{epsilon} samples from a standard normal,
\texttt{transform} would be $\mu + \sigma \epsilon$,
\texttt{gradTransform} would be $[1.0, \epsilon]$, and
\texttt{sampleGradOfLogQ} would be $-(z - \mu) / \sigma ^ 2$. To
reiterate, defining these functions for a variational factor is not
necessary for using a factor in \texttt{propagator-bbvi}, but they do
allow one to calculate gradients with the reparameterization
gradient. As we will see next, one of the strengths of
\texttt{propagators-bbvi} is that it allows us to easily mix and match
reparameteriation updates with score updates on parts of the model
that are or are not differentiable, and on factors for which we can or
cannot implement the functions in the \texttt{Differentiable}
typeclass. This approach lends itself to the goal of optmiziation
flexibility: we can use reparameterizations gradients (with their
lower variance) on factors where possible, and score gradients
elsewhere.

\subsection{Propagators: calculating gradients}

Propagators calcualte gradients by watching variational factor cells,
and then writing new gradients. The gradient calculations themselves
follow the procedures outlined in \citet{ranganath-2014} and
\citet{kuc-2017}, and are not novel. However how they are embedded in
the propagator framework does allow for development of new algorithms,
so I focus discussion on that. Each propagator calculates the gradient
using observations, variational cells, and a set of factors from the
log likelihood (e.g. the model specification). The number of factors
from the log likelihood that the propagator calcualtes a gradient on
is used in conjunction with the total number of factors stored in the
invariant data of the variational factor to re-weight the prior and
entropy terms when calculating the gradient update. For example, if
the latent variable corresponding to a variational cell is in 10
factors of a likelihood function, but a propagator only calculates the
gradient based on one factor, then the prior and entropy terms are
scaled down by $1/10$. Haskell's lazy evaluation also helps the user
implement flexible, distributed propagation. The user can specify the
necessary likelihood calculations on the full model, and use this same
specification in very local propators (for example, a gradient
propagator update for a single factor). When the propagator writes its
gradient to the variational cell, lazy evaluation insures that only
the pieces of the full likelihood relevant to the gradient propagation
are actually calculated. So, one can use the full likelihood
specification in local propagators without fear that uneeded likelhood
computaitons whill be execture. This all allows us to break up the
update computation into many different propagators, whcih can all push
updates to variational cells asynchronously. These design decisions
aid scalability.

% discuss rao-blackwellization here?

Beyond just the gradient calculation aspect of propagators, we can
also specify how they react to changes in their input cells and
leverege this to rapidly develop new update schedules. For example, it
is easy to specify that a propagator watches one variational factor
for a change, and then upon that change optmizes another variational
factor for many steps (or until covergence) with the watched
variational factor fixed. This flexibility in the manner in which
graident updates are applied to cells is indepndent of the caluclation
of the updates themselves, but allows users to rapidly test different
optmiziation orderings than those presented in the literature, which
could yield more rapid convergence depending on the model.

\section{Discussion}

\texttt{propagator-bbvi} allows easy specification of completely novel
optmiziation strategies for black box variational inference. Users can
mix and match reparameterization gradient updates with score gradient
updates, they can update variational factors from very local
(e.g. single factor) computations, they can specify different step
size functions and parameters for different factors, and they easily use
piecewize optimization where some variational factors are optimized to
arbitrary convergence while all others are held fixed. Initial tests
with the library suggest that optmizing variational factors with
others held fixed can increase convergence rates. Given that this
behavior is likely model-dependent and most of my thought has gone
into the library design and development of BBVI algorithms using
propagators, this report uses available its available space to discuss
design.

Additionally \texttt{propagator-bbvi} is easily extensible to new
variational factors and models. Introduction of a new variational
factor only requires the definiton of six functions for that factor,
three of which are trivial. Finally, \texttt{propagtor-bbvi} has the
ability to scale up to massively parallel computation. All graidne
tupdates, which can be as local as updateing a vairational factor
based on a single factor in the likelhoood, are evaluated
asycnronously, allowing them to be arbirarily parrelelized by the
scheduler. Currently, \texttt{propagator-bbvi} is build on the
\texttt{propagators} library which has a very simple serial scheduler
build on the ST monad. While still fast for serial computations,
really scaling up \texttt{propagator-bbvi} requires a more
sophisticated parralel scheduler, that also handles greedy propagator
specifications that in the current scheduler causes an exponential
space leak of thunks (discuss this?). The good news is that developing
new underlying schedulers is completely independent of the
\texttt{propagator-bbvi} code: now that BBVI has been framed in terms
of propagators, any executions scheduler can be used without changing
the propagator code.

In summary \texttt{propagator-bbvi} is an extensible, scalable
playground for developing flexible and nobel optmiziation strategies
for BBBI. I will continue to hack on it, extending it as needed by my
reaserch, and I hope others do as well.

\bibliography{references}

\end{document}
%%% Local Variables:
%%% mode: latex
%%% TeX-master: t
%%% End:
