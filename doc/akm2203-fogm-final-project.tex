\documentclass[12pt]{article}

% fonts
\usepackage[scaled=0.92]{helvet}   % set Helvetica as the sans-serif font
\renewcommand{\rmdefault}{ptm}     % set Times as the default text font

% db/ not mandatory, but i recommend you use mtpro for math fonts.
% there is a free version called mtprolite.

% \usepackage[amssymbols,subscriptcorrection,slantedGreek,nofontinfo]{mtpro2}

\usepackage[T1]{fontenc}
\usepackage{amsmath}
\usepackage{amsfonts}

% page numbers
\usepackage{fancyhdr}
\fancypagestyle{newstyle}{
\fancyhf{} % clear all header and footer fields
\fancyfoot[R]{\vspace{0.1in} \small \thepage}
\renewcommand{\headrulewidth}{0pt}
\renewcommand{\footrulewidth}{0pt}}
\pagestyle{newstyle}

% geometry of the page
\usepackage[top=1in,
            bottom=1in,
            left=1in,
            right=1in]{geometry}

% paragraph spacing
\setlength{\parindent}{0pt}
\setlength{\parskip}{2ex plus 0.4ex minus 0.2ex}

% useful packages
\usepackage{natbib}
\bibliographystyle{dinat}
\usepackage{epsfig}
\usepackage{url}
\usepackage{bm}

\usepackage{titlesec}
\titleformat*{\section}{\normalsize\bfseries}
\titleformat*{\subsection}{\normalsize\bfseries}

\usepackage{enumitem}
\setlist{nolistsep}

\begin{document}

\begin{flushleft}
\textbf{Propagator BBVI: a flexible framework for rapid design of BBVI
  optimization algorithms} \\
Adam Massmann (akm2203) \\
\today
\end{flushleft}

Here I present propagator BBVI: a BBVI inference approach built on top
of a propagator computing framework. Propagators consist of stateful
cells that accumulate information about a value, and autonomous
machines that add information about a value. In the case of BBVI,
cells are each variational factor, and the autonomous machines are the
computations that calculate gradients from the joint and update the
variational families. This framework allows us to rapidly develop and
test many (possibly novel) stochastic gradient update schemes: each
variational factor can use different step size functions and
convergence specifications, and we can mix both re-parameterization
and score-based gradient estimates on portions of the model that are
differentiable or not. Additionally, the propagator based framework
has the potential to scale to massively parallel computation,
implicitly with no necessary specification from the user. I use
propagator BBVI to test the convergence properties of different
algorithm designs: specifically I examine the usefulness of
iteratively optimizing some portions of the posterior with others held
fixed (e.g. a sort of black box CAVI). Propagator BBVI allows the user
to easily and rapidly specify many such algorithm designs.


\section{Introduction}

% goals of library
Box's loop allows us to interatively refine assumptions about our
data. However, efficient movement around the loop requires that
computationally tractable inference algorithms are easy to derive from
a model encoding our assumptions. Black box variational inference
(BBVI) performs inference on a very broad set of probablistic models
without difficult model-specific derivations, and is a path towards
quick iterations over Box's loop \citep[e.g.,][]{ranganath-2014,
  kuc-2017}. Here I build a new approach to BBVI upon a propagator
computing framework. The propagator-bbvi Haskell library aims for:

\begin{itemize}
\item A balance between the ease of specification and the flexibility
  of the inference optmizition: the probablistic model should be easy
  to specify on the computer and BBVI inference should follow
  automatically from the specification. Howev er, the library is also
  designed around easy user modification and performance tuning of the
  optmization algorithms underlying the inference methods. So, there
  is a tradeoff between automatic inference from the model
  specification and the flexibility to tune the inference algorithm
  for tractability. Here we favor a balance between adding complexity
  in what the user must specify in order to run inference, and
  granting the user the ability to try many different optimization
  strategies underlying inference. Sensible defaults can mitigate
  someof the added complexity of user specification.
\item Extensibility: the library should be easy for users to
  extend. For example, adding new variational distributions or
  optimization algorithms should be simple and additive to the
  codebase: no need to rewrite or modify existing code. To achieve
  this goal, we favor a shallow library structure rather than a more
  deeply embedded domain specific language (DSL). The benefits of a
  shallow library are that it gives users more flexibility for
  extension and specification through a designed blending between the
  library and exisitng language abstractions (in this case Haskell)
  . However, the downsides are that a shallow embedding requires that
  users may need a deeper understanding for how the library works
  internally, as well as familiarity with Haskell. A library approach
  could be a nice alternative to popular general inference tools like
  Stan and Infer.NET that favor more of a DSL approach (but note I am
  not as familiar with tools like Edward and Pyro; they may take a
  different approach).
\item Ability to scale to massive data and models: inference methods
  should scale to massive data. In addition to previous work on
  stochastic gradient methods to scale to massive data
  \citep[e.g.,][]{hoffman-2013, ranganath-2014, kuc-2017}, we hope to
  build a framework that easys parallelization of inference
  algorithms. In keeping with the extensibility goal, development of
  new parallel scheduling and data subsampling techniques possible
  without significant modification of the larger codebase, most
  importantly the machinery for specifiying models.
\end{itemize}

As I describe the library design and its novelty for developing new
approaches to BBVI, I will refer back to these three goals:
optmiziation flexibility, extensibility, and scalability.

\section{Introduction to propagators}

The propagator computing framework \citep{radul-2009} consists of
stateful cells that accumulate information about a value and stateless
autonomous propagators that do computations and add information to
cells. In other words, cells do no computation and propagators store
no data. The user attaches propagators to cells. Their are two
attachment types for propagators: \textit{watch} and \textit{write}
attachments. When cells acquire new information, they trigger any
propagators watching them to execute their computations with the new
data, and those propagators will write the result of their
computations to any cells they have a \textit{write} attachment to. If any
of these writes are new information for a given cell, then they will
trigger their own watching propagators to fire. In this manner,
propagators and cells form an asycnrounous computation network that
runs until no cells in the network can aquire any more information
(``quiescence'').

The propagator model is useful in many computational applications,
including functional reactive programming, dependency directed search,
and constraint satisfaction. For a full discussion I recommend
\citet{radul-2009}. However, why propagators for BBVI? In short, they
are a path towards massively parallel computation. Because all
computation is asyncronous and there is no global state, the ordering
and scheduling of computation is completely independent of the
propagator network. So, once a problem is coded in terms of
propagators, one can alter the scheduling of the propagator
computations to suit their goals without touching the propagator
network. In this way propgators directly lend themselves to my design
goals: (1) \textit{scalability} through parralel comutation,
and (2) \textit{extensibility}: computation scheduling is independent
of the algorithm code, so new parallel schedulers can be developed
without modification of any other pieces of the library.

\section{BBVI with Propagators}

Variational factors are represented as cells: they accumulate
information moving them closer (on average) to the true
posterior. Computations calculating gradients and updating the
variational factors are the propagators. A deeper dive into the design
will illuminate the consequences and benefits of this new approach to
BBVI. The library is publically available at:
\url{https://github.com/massma/propagator-bbvi}, and some sparse (but
growing) documentation is available at:
\url{http://www.columbia.edu/~akm2203/propagator-bbvi-0.1.0.0/}.

\subsection{Cells: variational factors}


\section{Discussion}

\section{Future Work}

\bibliography{references}

\end{document}
%%% Local Variables:
%%% mode: latex
%%% TeX-master: t
%%% End:
